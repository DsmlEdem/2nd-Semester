\documentclass{article}

\usepackage[utf8]{inputenc}
\usepackage{amsmath, amsfonts, amssymb}
\usepackage{amsthm}

\usepackage{listings}
\usepackage{xcolor}

\definecolor{codegreen}{rgb}{0,0.6,0}
\definecolor{codegray}{rgb}{0.5,0.5,0.5}
\definecolor{codepurple}{rgb}{0.58,0,0.82}
\definecolor{backcolour}{rgb}{0.95,0.95,0.92}

\lstdefinestyle{mystyle}{
    backgroundcolor=\color{backcolour},   
    commentstyle=\color{codegreen},
    keywordstyle=\color{magenta},
    numberstyle=\tiny\color{codegray},
    stringstyle=\color{codepurple},
    basicstyle=\ttfamily\footnotesize,
    breakatwhitespace=false,         
    breaklines=true,                 
    captionpos=b,                    
    keepspaces=true,                 
    numbers=left,                    
    numbersep=5pt,                  
    showspaces=false,                
    showstringspaces=false,
    showtabs=false,                  
    tabsize=2
}

\lstset{style=mystyle}

% To make custom snippets, Ctrl+shift+P --> Preferences: Configure User Snippets -> latex (or latex.json)
\DeclareMathOperator{\E}{\mathrm{E}}
\DeclareMathOperator{\V}{\mathrm{V}}
\DeclareMathOperator{\Var}{\mathrm{Var}}
\DeclareMathOperator{\Cov}{\mathrm{Cov}}

\newtheorem{exercise}{Exercise}[section]

\title{Data Driven Models for Engineering Problems\\Exercise 0}
\author{Konstantinos Papadakis\\DSML 03400149\\k.i.papadakis@gmail.com}
\date{March 2022}


\begin{document}

\maketitle


\section{Exercises}

Let \(X(t_n)\) be a random walk where each step is either 1 or -1 with equal probability.

% EXERCISE A
\begin{exercise}
    \( \E[X(t_n)] = 0 \)
\end{exercise}
\begin{proof}
    Let \( \xi(t_1), \dots , \xi(t_n) \) be the steps up to \(t_n\).
    By assumption, these are independent and take values
    in \(\{-1, 1\}\) with equal probability \(\frac{1}{2}\).
    \begin{equation*}
        \E[\xi_n] = (-1) \cdot P(\xi_n = -1) + 1 \cdot P(\xi_n = 1) = (-1) \cdot \frac{1}{2} + 1 \cdot \frac{1}{2} = 0
    \end{equation*}

    We have \(X(t_n) = \xi(t_1) + \dots + \xi(t_n) \), therefore
    \begin{equation*}
        \E[X(t_n)] = \E[\xi(t_1)] + \dots + \E[\xi(t_n)] = 0 + \dots + 0 = 0
    \end{equation*}
\end{proof}


% EXERCISE B
\begin{exercise}
    \( \V[X(t_n)] = n \)
\end{exercise}
\begin{proof}
    \begin{equation*}
        \begin{split}
            \V[\xi(t_n)] &= \E[\xi(t_n)^2] - \E[\xi(t_n)]^2 = \E[\xi(t_n)^2]\\
            &= (-1)^2 \cdot P(\xi_n = -1) + 1^2 \cdot P(\xi_n = 1)\\
            &= 1 \cdot \frac{1}{2} + 1 \cdot \frac{1}{2} = 1
        \end{split}
    \end{equation*}
    Because \(\xi(t_n)\) are independent, we have that
    \begin{equation*}
        \V[X(t_n)] = \V[\xi(t_1)] + \dots + \V[\xi(t_n)] = 1 + \dots + 1 = n
    \end{equation*}
\end{proof}


% EXERCISE C
\begin{exercise}
    The process is not ergodic.
\end{exercise}
\begin{proof}
    Let \(\hat\mu_X := \frac{1}{T} \sum_{i=0}^{T} X(t_n)\) be the time average estimate.
    We will show that this random variable does not converge
    to the (constant) ensemble average \(\mu_x = 0\).

    \begin{equation*}
        \E[\hat\mu_x] = \E\left[\frac{1}{T}\sum_{i=0}^{T} X(t_n)\right] = \frac{1}{T}\sum_{i=0}^{T} \E[X(t_n)] = 0
    \end{equation*}

    This shows that the time average estimate's mean is equal to \(\mu_x\), but as we shall prove,
    it does not converge to a \emph{constant}.

    \begin{equation*}
        \begin{split}
            \V[\mu_X] &= \E[\mu_x^2] - \E[\mu_x]^2 = \E[\mu_x^2]\\
            &= \frac{1}{T^2} \E\left[ \left( \sum_{i=1}^T X(t_i )\right)^2 \right]\\
            &= \frac{1}{T^2} \E\left[ \sum_{i=1}^T \sum_{j=1}^T X(t_i) X(t_j) \right]\\
            &= \frac{1}{T^2} \sum_{i=1}^T \sum_{j=1}^T \E[X(t_i) X(t_j)]\\
            &= \frac{1}{T^2} \sum_{i=1}^T \sum_{j=1}^T \Cov[X(t_i), X(t_j)] + \E[X(t_i)] \E[X(t_j)]\\
            &= \frac{1}{T^2} \sum_{i=1}^T \sum_{j=1}^T \Cov[X(t_i), X(t_j)]\\
        \end{split}
    \end{equation*}

    Since \(\xi(t_i)\) are independent and with variance 1, we have that

    \begin{equation*}
        \begin{split}
            \Cov[X(t_i), X(t_j)] &= \Cov[\sum_{r=1}^i \xi(t_r), \sum_{s=1}^j X(t_s)]\\
            &= \sum_{r = s} \Cov[\xi(t_r), \xi(t_s)] + \sum_{r \neq s} \Cov[\xi(t_r), \xi(t_s)]\\
            &= \sum_{k = 1}^{\min\{i, j\}} \Cov[\xi(t_k), \xi(t_k)]\\
            &= \sum_{k = 1}^{\min\{i, j\}} 1\\
            &= \min\{i, j\}
        \end{split}
    \end{equation*}

    Therefore,

    \begin{equation*}
        \begin{split}
            \V[\mu_X] &= \frac{1}{T^2} \sum_{i=1}^T \sum_{j=1}^T \min\{i, j\}
        \end{split}
    \end{equation*}
    
    We now show that \(\sum_{i=1}^T \sum_{j=1}^T \min\{i, j\} = \sum_{k=1}^T k^2\).
    Evaluate the summands of the left hand side in the following table:

    \begin{equation*}
        \begin{matrix}
            1&1&1&1&\ldots&1\\
            1&2&2&2&\ldots&2\\
            1&2&3&3&\ldots&3\\
            1&2&3&4&\ldots&4\\
            \vdots&\vdots&\vdots&\vdots&\ddots&\vdots\\
            1&2&3&4&\ldots&T
        \end{matrix}
    \end{equation*}

    We can view this table as stacked 3-dimensional unit cubes.
    The number at position \((i, j)\) represents the number of cubes that are stacked vertically at this position.
    At height 1 we have \(T^2\) cubes, at height 2 we have \((T-1)^2\) cubes, \(\dots\), at height \(T\) we have \(1^2\) cubes.
    This is proves the equality visually.

    We now have

    \begin{equation*}
        \begin{split}
            \V[\mu_X] &= \frac{1}{T^2} \sum_{i=1}^T \sum_{j=1}^T \min\{i, j\}\\
                      &= \frac{1}{T^2} \sum_{k=1}^T k^2\\
                      &= \frac{1}{T^2} \frac{T (T+1) (2T+1)}{6}\\
                      &= \frac{1}{6} (1 + \frac{1}{T}) (2T + 1) \xrightarrow{T \to \infty} \infty
        \end{split}
    \end{equation*}

    where we used the well-known formula for the sum of squares of the first n natural numbers.

    This shows that the variance of \(\mu_X\) does not become 0,
    and thus \(\mu_X\) doesn't converge to a constant. 
\end{proof}


\section{Code}

We can also write Matlab code that attempts to estimate the values proved above.

\lstinputlisting[language=Matlab, caption={Matlab simulation}, label={lst:matlab-simulation}]{exercise_0.m}


\end{document}